\title{A Very Simple \LaTeXe{} Template}
\author{
        Vitaly Surazhsky \\
                Department of Computer Science\\
        Technion---Israel Institute of Technology\\
        Technion City, Haifa 32000, \underline{Israel}
            \and
        Yossi Gil\\
        Department of Computer Science\\
        Technion---Israel Institute of Technology\\
        Technion City, Haifa 32000, \underline{Israel}
}
\date{\today}

\documentclass[14pt]{article}

\begin{document}
\maketitle

\begin{abstract}
This is the paper's abstract \ldots
\end{abstract}

\section{Introduction}
Correctness of computer systems is important in our software dependent society. We depend more and more on this modern computer systems. Such systems consist of complex hardware and software components. So correctness of both hardware and software is important. However, it is much harder to ensure the correctness of the software part of these systems. As the complexity of the software parts are getting more complex than that of the underlying hardware.  And manual inspection of complex software is error-prone and costly.  Numerous formal tools to find functional design bugs in hardware are available and in wide-spread use.  In contrast, the market for software verification tools is still in its infancy. A lot of research in this field is going on. We need highly automated method that provides rigorous guarantee of quality. These methods should be scalable enough to match the enormous complexity of software systems. Bounded model checking (BMC) of programs is one of such methods used for software verification. In this paper we will explore LLBMC, an implementation of the idea of  Bounded model checking.

\paragraph{Outline}
The remainder of this article is organized as follows.
Section~\ref{previous work} gives account of previous work.
Our new and exciting results are described in Section~\ref{results}.
Finally, Section~\ref{conclusions} gives the conclusions.

\section{Verification problem}\label{verification problem}
In industry we see the use of testing to ensure software quality and even to find bugs. However such quality guarantees do not reflect to ?correctness? of  the system. As we know that ensuring the absence of all errors in a design is usually too expensive. So all testing based approaches e.g. random testing and automated test-case generation are incomplete.  The other approach depends on the fact that systems can be viewed as mathematical objects with well-specified behaviour. Or we can specify the system ( intended behaviour) using mathematical logic. Then one can reason about whether the system meets its specification or not. This field of study has been active and It is often referred to as formal methods. The \textit{verification problem} is: Given program M and specification h determine whether or not the behaviour of \textit{M} meets the specification \textit{h}.  The specification is formulated in mathematical logic. The model checking problem is an instance of the verification problem. Model checking provides an automated method for verifying finite state systems and checks it for certain properties. The method is algorithmic and often efficient because the system is finite state. If the check fails, that means the design has a bug. And, most of the time, the model checkers generate a counterexample how the bug is produced so the developer can easily figure out what is the mistake.

\section{Previous work}\label{previous work}
A much longer \LaTeXe{} example was written by Gil~\cite{Gil:02}.


\section{Results}\label{results}
In this section we describe the results.

\section{Conclusions}\label{conclusions}
We worked hard, and achieved very little.

\bibliographystyle{abbrv}
\bibliography{simple}

\end{document}
This is never printed
